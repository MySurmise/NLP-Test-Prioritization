\section{Evaluation}
\subsection{Research Questions}
To test our approach, we want to use our dataset of XWiki to answer the following  \acp{rq}:
\begin{itemize}
    \item \ac{rq}1: Does clustering test cases by test steps cluster similar tests?
    Each test consists of a number of test steps to execute when executing the tests. It is not given that clustering these test steps actually gives us a clustering that clusters similar test cases together.
    \item \ac{rq}2: How much time/many tests can be saved by selecting tests based on clustering test cases by test steps?
    The goal of this paper is to demonstrate wether clustering tests in said method can produce helpful results. But is there actually a substantial amount of time/tests that can be saved when only selecting a portion of all tests?
    \item \ac{rq}3: Does the proposed method pose a threat to fault coverage if only a subset of tests are executed?
    \item \ac{rq}4: What is an appropriate amount of tests to be executed without threatening the fault coverage too much?
    \\\\\indent{alternatively (shorter):}
    \item \ac{rq}1: Does clustering test cases by steps effectively group similar tests and reduce test count or time?
    \item \ac{rq}2:Does clustering test cases by steps effectively reduce test count or time?
    \item \ac{rq}3: What balance ensures optimal fault coverage with minimum tests after clustering?
\end{itemize}